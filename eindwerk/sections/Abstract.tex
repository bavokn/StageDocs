A single drone is not able to accomplish much on a large scale. Compare this to a bee trying to harvest a 
field of flowers, an impossible task for just a single bee. Nature solves this problem using swarm intelligence, 
using a colony of bees working together to solve a bigger problem. This phenomenon is not only perceived in bees. Birds flocking, 
hawks hunting, animal herding, bacterial growth, fish schooling and microbial intelligence are all examples 
of swarm intelligence in nature. Multi-agent \acs{uav} systems are based on swarm intelligence, using multiple 
\acsp{uav} (aka drones) to solve more complex and bigger problems that are impossible for just a single \acs{uav} to solve.

This research is conducted by PXL Smart ICT, the IT \& Electronics center of expertise and part of the research 
department at PXL University of Applied Sciences and Arts. The goal of the project is to enable IT companies to 
implement \acs{uav} projects via rapid robot prototyping.

The internship provides a multi-agent system (\acs{mas}) architecture that supports homogeneous as well as heterogeneous 
structures. Each \acs{uav} is an agent that processes data on its own and shares its knowledge with the other agents. 
The agent organization can handle hierarchical, hologenic and coalition teams. Different types of organizations will 
be compared regarding task completion time, computational time and accuracy of the completed task. A blackboard system 
takes care of knowledge sharing between agents, with the option of working over multiple computers. 

\acs{ros} (Robotic Operating System) is used to control the \acsp{uav}, each equipped with a PX4 autopilot. \acs{ros} is a set of 
software libraries and tools that help to build smart robot applications. The PX4 autopilot uses MAVLink, a very 
lightweight messaging protocol for communicating with \acsp{uav}. MAVROS, an implementation of MAVLink in \acs{ros}, is a 
component that can convert between \acs{ros} messages and MAVLink messages. This provides easy control of the \acsp{uav}. By 
gathering information from the \acs{uav} components the agents can analyze data. For example, the agent receives images 
from the camera and can detect if an object is a car using OpenCV, a person, or something else.

This project shows how to create a flexible multi-agent system (\acs{mas}) architecture including a few exemplary 
demonstrators. In these examples, it is made clear how each \acs{uav} makes decisions and how data is shared. 
With each example, the flight patterns are plotted to visualize changes in decision-making in other architectures.
