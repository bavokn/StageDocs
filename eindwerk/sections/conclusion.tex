Concluding the project, three different multi-agent architectures are constructed and demonstrated. With
clarification and data plotting on all three architectures it becomes clear on how these
architectures differentiate. Each architecture example has its own start script.

The agents can communicate with each other using the blackboard system. 
In the blackboard system an area can be divided into multiple subsections. In each 
subsection, tasks can be assigned. Using the communication the multi-agent system can solve the tasks given.
Depending on the architecture the agents will autonomously solve the given tasks in a
specific order.

As an example of how to add extra components, an object detector is made. Showing on
how to add a component to the system and linking this to the blackboard system. The
detector can be run on the \acs{gpu} but unfortunately not in a multiprocess environment.

A multi-agent system can be run on different machines using a docker swarm. The scripts to start an
join a swarm are inside the project in a separate folder named \textit{Docker\_swarm\_scripts}.