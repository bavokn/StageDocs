One of the biggest cons with off-the-shelve multi-agent architectures is that they are almost
always designed for a specific research topic. The interactions with the environments are
not generic and prototyping is very difficult because of the limitations in setting the goals
of the intelligent agents. The data sharing between agents is often immutable and therefor
adding sharable knowledge is impossible.

An advantage of using an existing architecture is that the research is already conducted
and the limitations found by the developer. Knowing the limits of an architecture can help
optimize problem-solving by knowing its optimal performance setup.