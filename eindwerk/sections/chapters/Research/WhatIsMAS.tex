
Like the definition of an agent, there has yet to be a true definition for a multi-agent system. 
Yet again there are a lot of definitions, each depending on the background of the research they are conducting.
 For this research the definition is chosen from P. Stone. By \citet{mas:def}, a multi-agent is defined as:

\begin{center}
    \textit{\textbf{"A multi-agent system is a loosely coupled network of problem-solving entities (agents) that work together to find answers to problems that are beyond the individual capabilities or knowledge of each entity (agent)."}}
\end{center}

 
Per definition a multi-agent system is a group of agents organized to solve a problem a single agent would not be able to solve on its own. 
The agents are organized, they have a structure and are able to communicate with each other to share knowledge. 

An agent in a multi-agent system has some important characteristics. As mentioned before, the agent must be autonomous in some form. 
It can make decisions on its own without any intervention. Secondly, no agent has a full view of the world. And thirdly the system must be decentralized. 
There is no controller controlling each agent independently. 

There can be different types of internal hierarchies in a multi-agent system, depending on how the developer wants the agents to work together.
