
MAVLink is a very lightweight messaging protocol for communicating with air vechicles. Data streams are sent/published as topics.  
The messages are defined within \acs{xml} files. MAVLink is very efficient, it does not require additional framing and is very well 
suited for projects with limited bandwidth. Next to being very efficient, MAVLink is also very reliable. It provides methods for 
detecting corrupt packages, if any packets have been dropped, and for authenticating packets. A developer is able to control 255 concurrent 
systems on the network when using MAVLink, with each system having the ability for onboard and offboard communication \cite{mavlink:about}.

\begin{figure}[ht]
    \centering
    \includegraphics[scale=0.8]{mavlink.png}
    \caption[MAVLINK logo]{MAVLINK logo}
\end{figure}

An example of a mavlink message is a HEARTBEAT message. A vechicle must regularly broadcast their HEARTBEAT to make sure it is still connected. 
A developer can set the rate at which the message is broadcasted, and how many of these messages may be missed before a vechicle is labbeled missing 
or disconnected \cite{mavlink:heartbeat}.

Many languages support MAVLink v1, MAVLink 2 and Message Signing. There are also prebuilt MAVLink C libraries available on the MAVLink website.

