
Each \acs{uav} is an agent. To create an agent in \acs{ros}, firstly a node is created. 
A node is an executable that uses \acs{ros} to communicate with other nodes \cite{ros:nodes}. For this small example the name \acs{uav}\_0 is given to the node.

\begin{verbatim}
   rospy.init_node('UAV_0', anonymous=True)
\end{verbatim}

The \acs{uav} is able to situate itself using an internal \acs{gps}. 
Using MAVROS the developer is able to control a \acs{uav}.  
MAVROS will translate the messages sent from \acs{ros} via MAVLink so the \acs{uav} can “understand” these messages. 
To send a message the developer must first create a publisher to a specific topic, in this example the \acs{uav} is sent to a specific location.

\begin{verbatim}
   point_push = rospy.Publisher('/iris_0/mavros/setpoint_raw/local',
                                       PositionTarget, queue_size=1)
\end{verbatim}

A publisher is made for the topic ‘/iris\_0/mavros/setpoint\_raw/local’. When publishing to this point the data must be a PositionTarget. 
Each topic has a defined type of message so that MAVROS can translate this message to MAVLink. Following the same reasoning multiple publishers 
can be made for more complex movements or actions. Next to creating more publishers a developer also has the ability to create custom topics with custom messages. 

A \acs{uav} can also receive information with a subscriber. Like in a publisher the type of message has to be defined. 
Finally a callback has to be given as an argument where the data is processed. These callbacks are default run single-threaded. 
This can cause some performance issues in bigger problems. To solve this problem \acs{ros} provides Multi-threaded spinning and AsyncSpinner \cite{ros:callbacks}. 
For this short example, processing a local \acs{gps} message, a subscriber is needed with the callback. 

\begin{verbatim}
   gps_local_sub = rospy.Subscriber('iris_0/mavros/global_position/local',
                                   Odometry, self.gps_local_callback)
\end{verbatim}

\newpage
The subscriber receives messages from the topic ‘iris\_0/mavros/global\_position/local' where the message type is Odometry. 
The callback, self.gps\_local\_callback , will process the data each time it is updated. 

\begin{verbatim}
   def gps_local_callback(self, info):
      pose = info.pose
      self.x = pose.pose.position.x
      self.y = pose.pose.position.y
      self.z  = pose.pose.position.z
\end{verbatim}

In this case the local position of the \acs{uav} is updated each time the callback is called. Multiple callbacks can be made in a single node to further 
improve the complexity and possibilities of the agent. 