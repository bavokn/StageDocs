A docker swarm is a group of virtual or physical machines operating together in a cluster.
In the cluster, there is a master where other machines can join as workers. This master
nodes functions as the manager node, with the primary function being assigning tasks to
worker nodes [2].

In the project, there is a subfolder with scripts to construct a docker swarm. For \acs{ros} nodes
to be able to communicate with each other, they need to be connected to the same \acs{ros}
master. When executing the script \textit{0a\_create\_docke\_rswarm.sh} a docker swarm is initialized.
With the swarm initialized the script automatically constructs an overlay network called
\textit{my\_ros\_overlay\_network}. This overlay will be passed as network arguments when starting
a new docker container. These containers will then be attached to the overlay network.
When the script completed initializing the swarm and the overlay network it will return a
command to join the swarm.

A container is attached to the network when the network argument is passed with the
overlay network :

\begin{verbatim}
    --network=my_ros_overlay_network
\end{verbatim}

A linked volume is created for the container to be able to access generated files later.

\begin{verbatim}
    -v `pwd`/../Commands/bin:/home/user/bin \
\end{verbatim}

On the master node the simulation is started with the script \textit{2\_start\_simulation\_attached.sh}. This will 
start a container attached to the network. before starting the simulation, the \textit{ROS\_MASTER\_URI} has to be 
set accordingly. \textit{0c\_generate\_ros\_config\_command.sh} first checks if the worker is the master or not. When 
a machine is the master the IP address is set to \textit{localhost}.

\begin{verbatim}
IP=$(docker inspect -f 
'{{range .NetworkSettings.Networks}}{{.IPAddress}}{{end}}' 
smartuav_simulation_attached) > ../Commands/bin/set_swarm_settings.bash

echo "export ROS_HOSTNAME=$IP" >> ../Commands/bin/set_swarm_settings.bash 
echo "export ROS_IP=$IP" >> ../Commands/bin/set_swarm_settings.bash
echo "export ROS_MASTER_URI=http://localhost:11311" 
>> ../Commands/bin/set_swarm_settings.bash
\end{verbatim}

These variables are saved in a file \textit{set\_swarm\_settings.bash} and can now be sourced from inside the container. 
\newpage

To enable workers to communicate with the \acs{ros} master, on the master node of the swarm, the overlay \acs{ip} address 
has to be known. The script \textit{0b\_generate\_overlay.sh}, on the master node
(outside of running container), will return the 
\acs{ip} address of the running container. 

On a worker machine, \textit{0c\_generate\_ros\_config\_command.sh} first checks if the machine is the host or not. When it 
is not the script will need two inputs; the \acs{ip} and the current running attached container. Instead of setting the IP to localhost, 
the script will now generate
the \textit{ROS\_MASTER\_URI} and the \textit{ROS\_IP} correctly for each worker.

\begin{verbatim}
    read -p "IP on overlay network of the master container: " master_ip
    read -p "Name of the running container: " running_container

    readarray -d _ -t strarr <<<"$running_container"

    dir=${strarr[1]}
    #empty out settings
    echo "" > ../src/$dir/Commands/bin/set_swarm_settings.bash 

    IP=$(docker inspect -f 
    '{{range .NetworkSettings.Networks}}{{.IPAddress}}{{end}}' 
    $running_container)
    
    echo "export ROS_HOSTNAME=$IP" 
    >> ../src/$dir/Commands/bin/set_swarm_settings.bash 

    echo "export ROS_IP=$IP" 
    >> ../src/$dir/Commands/bin/set_swarm_settings.bash  

    echo "export ROS_MASTER_URI=http://$master_ip:11311/" 
    >> ../src/$dir/Commands/bin/set_swarm_settings.bash
\end{verbatim}

The current running container name is needed to get the \acs{ip} of that container to set the variable \textit{ROS\_IP}. If this is not done 
the subribers would not be able to receive messages from this node.
The script will then automatically place the \textit{set\_swarm\_settings.bash} in the correct linked volume folder. This file can now be sourced
from inside the container.

To test if each worker is connected to the \acs{ros} master the simulation has to be 
started with \textit{1\_start\_simulation\_attached.sh}. This will start Gazebo, Rviz, and roscore. If the command 
\textit{rostopic list} from inside each worker container returns the topics, the worker is connected to the master 
and is able to communicate with it.