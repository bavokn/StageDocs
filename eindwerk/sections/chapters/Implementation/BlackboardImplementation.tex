
The second container, the blackboard system, enables each agent to communicate with eachother.
The container uses the image \textit{smartuav\_ros\_mavros}. MAVLink, MAVROS and pymavlink are installed in this image. 
Using these libraries a developer can construct custom messages that can be updated in the blackboard system. 
The blackboard can then publish these custom messages as a bundled package to a custom topic.
A problem can be placed inside this package with flags and actions for the agents.

When handling a problem, each agent will take a portion of the problem and iteratively update the blackboard node. 
For example searching for an object in a large field. Each agent will take a portion of that field and search for the object. 
When an agent finds the object it updates the blackboard, therefore notifying the other agents that the object is found \cite{blackboard:wiki}.

When implementing a blackboard system in the project it unlocks a whole new set of possibilities. 
\acs{uav} collision prevention between each other, for example. Next to just GPS information agents are now able to distribute a problem 
over multiple agents. 
Communication between agents introduces a multi-agent system.

The limitation of 10 \acp{uav} is in this container. To add information to the blackboard a subsriber has to be made for each \acs{uav}. 
unfortunatly each substriber callback needs to have its own function hard coded to be thread safe. If this is not done 
each callback would refer to the same function and it would scramble to blackboard data.
