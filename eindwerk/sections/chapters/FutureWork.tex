\subsubsection{Multithreading object detection}

In the current state of the project, a normal multiprocess is started with the function \textit{pool} from 
the multiprocessing module. OpenCV 4.3.0 built with \acs{cuda} 10.2 backend does not support this. To fix this problem
te library \textit{pytorch} has to be used. Pythorch is an open source machine learning library based on the 
Torch library, used for applications such as computer vision. In this library there also is a multiprocessing module
providing support for \acs{cuda}-10.2. By running the object detection on the \acs{gpu} as a multiprocess it processes up to 500x faster.

\subsubsection{Optimizing docker swarm}

When currently starting a swarm a lot of scripts have to be manually executed. One of the optimizations would be to automize this using 
python. The python script should generate all the files needed and distrubute them over the docker containers using the overlay network. 
The user would no longer execute every script on every worker, thus reducing the chances of failing.

When running the docker swarm on a \acs{lan} connection it may fail. This is because some ports refuse connection 
from other machines. When this happens the system will throw an \textit{IOError} in the controller container.

\subsubsection{Adding other components}

Currently, there are not a lot of additional components in the project. There is a possibility
to add different components using extra docker containers or using the existing ones. One
example of an extra component is the SLAM (Simultaneous localization and mapping)
algorithm. It can be added using an extra container. The code to run this in a multiprocess
environment still has to be written.

When adding new components that need to be run in the swarm the name has to be justified as: \textit{workspace\_linkedvolumefoldername\_*}.
It is very import that the linkedvolumefoldername is correct because the script \textit{0c\_generate\_ros\_config\_command.sh} splits this string and 
takes the second argument to place the  \textit{set\_swarm\_settings.bash} in the correct folder.

\subsubsection{Implementing a decision making model}

An agent decides its task purely on what is available in its architecture. It can be optimized
by implementing a decision-making model in the multi-agent system. This would introduce
a whole other concept on itself. The model would need training and testing for different
scenarios. When a good model is implemented it would yield far better results than the
current decision making.