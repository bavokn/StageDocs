\subsection{Situatieschets stagebedrijf + motivatie}
  % In de situatieschets geef je aan in welke omgeving het project zich afspeelt. Omschrijf de aard van het stagebedrijf en de activiteiten van het stagebedrijf. Wat is het doel van de organisatie? Welke producten of diensten worden er geleverd? Welke zijn de Unique Selling Points van het bedrijf op vlak van IT (in welk IT domein zijn ze gespecialiseerd, op welk vlak van de IT business willen ze zich profileren, ...)? Wat is jouw motivering van het gekozen domein, het gekozen project en het gekozen stagebedrijf?
  The Internship company, the expertise center PXL smart ICT, develops all kinds of smart applications of emerging technologies and combines practical research and services at the intersection of IT and electronics. It’s located in Corda campus – Hasselt, central Limburg, Belgium. 

  Next to the smart ICT department PXL university also has a research team that yearly develop multiple projects. PXL university offers a wide variety of courses for students. 

  The purpose of next semester is to set up a multi agent UAV system for businesses/ people so that they have a solid start base to develop their own prototypes applications.
\newpage

\subsection{Probleemstelling(en)}
  % Waarom er iets moet worden geproduceerd, blijkt uit de probleemstelling. Wat is het probleem dat je moet gaan oplossen? Waarom werd het project opgestart, in welke fase bevindt het project zich op het moment van de stage,...? Wat er geproduceerd gaat worden, wordt in de doelstelling uiteengezet.
  Currently there is no framework available to easily prototype multi agent UAV systems in a simple environment. In case a company/person would like to develop such technologies they would need to start of from nothing which would mean they would first need to invest time/money into developing such environments.

The current state of the project is a proof of concept with a single UAV. Developed for firefighters to give live feedback from the air, scanning the area for possible dangerous chemicals.
\newpage

\subsection{Doelstelling(en)}
  % Door middel van een doelstelling wordt het gewenste eindresultaat van het project omschreven. Onder omschrijving verstaan we een bondige en relevante beschrijving van de doelstellingen die je wil verwezenlijken, de methodiek die je kan/gaat aanwenden, ...
  A flexible multi computer MAS architecture for robots. An implementation of the multi computer MAS architecture including a few exemplary demonstrators and a hierarchical state machine feature to make decisions. 

  For further research of the architectures there needs to be an in-depth comparison and analysis of different multi-agent decision-making systems.
\newpage

\subsection{Randvoorwaarden}
  % Welke afspraken heb je gemaakt met:
    % • het stagebedrijf, de bedrijfspromotor (werkregime, gedrag, …);
    % • de school, de hogeschoolpromotor (overlegfrequentie, manier van rapporteren, …);
    % • jezelf (vooropgestelde actiepunten, inzet, doel, …);
    % • de medestudenten indien je met meerdere studenten aan een project werkt.
  \begin{outline}
    \1 \textbf{Beslissingen:}
      \2[] Er worden veel technologiën uitgetest, in het resultaat wordt gekeken naar de snelheid hiervan, de robuustheid en de nauwkeurigheid. Met al deze resultaten in rekening gehouden gaat dan de beste gekozen worden voor verder gebruikt in het project
    \1 \textbf{Beperkingen:}
      \2[] Vrije tijd is een beperkende factor. Naast stage werk ik ook nog als jobstudent bij practinet. Daarnaast heb ik ook nog een extra opdracht voor I-Talent om wat uren in te halen die ik heb gemist omdat ik op Erasmus was.
    \1 \textbf{Kritische succesfactoren:}
      \2[] een workstation is een must bij deze opdracht. Er worden 5-10 UAVs gesimuleerd en hebben elk wel wat reken vermogen nodig. Op een gewone laptop zou dit onmogelijk zijn.
    \1 \textbf{Onzekerheden:}
      \2[] Het systeem werkende krijgen over meerde computers. Dit is moeilijk te testen zonder meerdere computers ter beschikking. Meerde architecturen ten volle kunnen testen.
    \1 \textbf{Afspraken:}
      \2 Studenten houden zich aan de glijdende werkuren van \stagebedrijf{} (I.e. 38u per week, starten $\leq$ 9 am, minimaal 30 min middagpauze)
      \2 Alles wordt ontwikkeld op eigen laptop of op de voorziene workstations, tijdens de werkuren pushen naar een met de begeleider afgesproken repository
        \3 Version Control verplicht te gebruiken voor Smart-ICT
        \3 Github repo te voorzien door begeleider
      \2 Voortgang bewaken door frequent stand-up meeting te doen
        \3 Dagelijks invullen in $\#$standup kanaal op Slack workspace vóór 09u15
        \3 Fysieke stand-up met alle stagairs $\rightarrow$ 1x per week op vrijdag
      \2 1x per sprint presentatie van resultaten naar de begeleider toe (sprint planning)
      \2 Insturingen
        \3 Portfolio’s en andere communicatie steeds via EPOS naar begeleider PXL 
        \3 Deadlines voor teksten op EPOS respecteren! 
        \3 \colorbox{yellow}{Geel markeren wat nieuw of aangepast is}
      \2 Solliciteren tijdens stageperiode
        \3 Flexibel
        \3 Afstemmen met begeleiders en team
        \3 Tijd inhalen
        \3 Andere afspraken uit PPT stagebegeleiding
      \2 Geen shenanigans
        \3 Geen speeltuin
        \3 Geen boksring
        \3 etc.
  \end{outline}

\newpage