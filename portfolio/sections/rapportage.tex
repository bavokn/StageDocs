\subsection{Wekelijkse rapportage}
  \begin{tabularx}{\textwidth}{| l | X |}
    \hline
    Datum: & 24/02/2020 - 28/02/2020\\
    \hline
    Geplande taken: &
    \begin{outline}
      \1 installatie workstations
      \1 opzetten dockerfiles
      \1 simulatie runnen met meerdere drones
      \1 klaarzetten voor implementaties
      \1 onderzoek voeren
    \end{outline}\\
    \hline
    Stand van zaken: & 
    \begin{outline}
      \1 dockerfiles + simumatie werken
      \1 meerdere drones tegelijk aansturen
    \end{outline}\\
    \hline
    Problemen en knelpunten: & 
    \begin{outline}
      \1 generieke oplossing zoeken voor implementatie
      \1 zoeken naar generieke oplossing om resulaten te vergelijken
      \1 pycharm werkt niet
    \end{outline}
    \\
    \hline
    Oplossingen: & VS code aan het gebruiken.\\
    \hline
    Persoonlijke reflectie: & de installatie verliep vlot dankzij Borcherd en Vic. Paar Roadblocks al overkomen \\
    \hline
    Planning volgende week: & de knelpunten oplossen\\
    \hline
  \end{tabularx}

  \newpage
  
  \begin{tabularx}{\textwidth}{| l | X |}
    \hline
    Datum: & 02/03/2020 - 06/03/2020\\
    \hline
    Geplande taken: &
    \begin{outline}
      \1 multithreading agents
      \1 methodes uitschrijven voor oppervlakte verkkening
      \1 controller voor makkelijke aansturing drones
      \1 blackboard node opzetten
      \1 custom messages
      \1 classe voor agents
      \1 hologenic structure
    \end{outline}\\
    \hline
    Stand van zaken: & 
    \begin{outline}
      \1 classe agents
      \1 controller werkt
      \1 methodes zijn uitgeschreven
      \1 blackboard node werkt
      \1 hologenic structure
    \end{outline}\\
    \hline
    Problemen en knelpunten: & 
    \begin{outline}
      \1 aansturing van de drones
      \1 generieke oplossing voor een makkelijke controller
      \1 uitschrijven methodes om oppervlaktes te verkennen
      \1 te grote foutmarge op global gps
      \1 aanpassen van gps coordinates van global naar local
    \end{outline}
    \\
    \hline
    Oplossingen: & 
    \begin{outline}
      \1 Issues gepost en oplossing gekregen
      \1 multithreading opzoeken voor controller
    \end{outline}\\
    \hline
    Persoonlijke reflectie: & Te lang blijven hangen op domme fout bij de aansturing van de drones \\
    \hline
    Planning volgende week: & drones zelf laten vliegen, hard gecodeerd op een gebied\\
    \hline
  \end{tabularx}

\newpage 

\begin{tabularx}{\textwidth}{| l | X |}
  \hline
  Datum: & 09/03/2020 - 13/03/2020\\
  \hline
  Geplande taken: &
  \begin{outline}
    \1 drone paden laten afvliegen
    \1 methodes uittesten van agent om zelf punten te genereren
    \1 camera implementatie
    \1 snelheid van drones aanpassen
    \1 slam implementatie
  \end{outline}\\
  \hline
  Stand van zaken: & 
  \begin{outline}
    \1 slam werkt maar overlapt → geen translatie op odom topic
    \1 UAVs vliegen op rustige snelheid
    \1 UAVs genereren zelf punten om af te vliegen
    \1 communicatie tussen agents, enkel coordinaten voorlopig
  \end{outline}\\
  \hline
  Problemen en knelpunten: & 
  \begin{outline}
    \1 aanpassen snelheid drones
    \1 communicatie tussen drones
    \1 zelf genereren van punten op map voor elke drone apart
    \1 drone die niks deed op publishen van een punt
  \end{outline}
  \\
  \hline
  Oplossingen: & 
  \begin{outline}
    \1 bug fix in de code
  \end{outline}\\
  \hline
  Persoonlijke reflectie: & Goede omvorming van code, onderzoek verliep vlot. Bronnen niet goed genoeg bijgehouden \\
  \hline
  Planning volgende week: & aanpassen algoritme zelf te vliegen. Meer gebieden dan drones zodat drones communiceren welk gebied verkend moet worden\\
  \hline
\end{tabularx}

\newpage 

\begin{tabularx}{\textwidth}{| l | X |}
  \hline
  Datum: & 16/03/2020 - 20/03/2020\\
  \hline
  Geplande taken: &
  \begin{outline}
    \1 ombouwen vliegalogritme
    \1 alle communicatie via blackboard node
    \1 drones communiceren om gebied te verkennen, geen overlap
    \1 custom messages met flags voor acties op gebieden
    \1 rtab implementeren
    \1 van xarco files naar sdf files
  \end{outline}\\
  \hline
  Stand van zaken: & 
  \begin{outline}
    \1 communicatie via blackboard node
    \1 dynamische toekenning gebieden aan drones
    \1 custom messages met flags voor acties op gebieden
    \1 van xarco files naar sdf files
  \end{outline}\\
  \hline
  Problemen en knelpunten: & 
  \begin{outline}
    \1 rtab implementeren
    \1 van xarco files naar sdf files -> kan geen meerdere drones aansturen
  \end{outline}
  \\
  \hline
  Oplossingen: & 
  \begin{outline}
    \1 terug naar xarco files
    \1 issue gepost maar geen reactie
  \end{outline}\\
  \hline
  Persoonlijke reflectie: & Veel te lang blijven hangen op bug in sdf files, issue gepost maar geen reactie \\
  \hline
  Planning volgende week: & object detection met opencv\\
  \hline
\end{tabularx}

\newpage 

\begin{tabularx}{\textwidth}{| l | X |}
  \hline
  Datum: & 23/03/2020 - 27/03/2020\\
  \hline
  Geplande taken: &
  \begin{outline}
    \1 projectomschrijving feedback verwerken
    \1 protfolio verder aanvullen
    \1 opencv object detection (mensen en autos)
    \1 onderzoek betere object detectors ( huidige : HOG, opencv haar)
    \1 bugfixes (opencv window crash)
    \1 nieuwe rtab map implementatie 
    \1 teamcontroller
  \end{outline}\\
  \hline
  Stand van zaken: & 
  \begin{outline}
    \1 Opencv geïmplementeerd
    \1 window crash na 1ste keer runnen controller → volledige restart voor nog eens te runnen ?
    \1 sdf terug naar xarco files voor de simulatie
    \1 translatie error op de rtab algoritme → mappen overlappen niet van verschillende drones
  \end{outline}\\
  \hline
  Problemen en knelpunten: & 
  \begin{outline}
    \1 translatie error op de rtab algoritme
    \1 xml files zijn niet goed
    \1 teamcontroller werkt nog niet , heeft veel userinput nodig
  \end{outline}
  \\
  \hline
  Oplossingen: & 
  \begin{outline}
    \1 andere classifier zoeken / color detection
    \1 research naar smash
  \end{outline}\\
  \hline
  Persoonlijke reflectie: & Het schrijven kan veel beter \\
  \hline
  Planning volgende week: & 
  \begin{outline}
    \1 object detection verfijnen
    \1 bugfixes
    \1 teamcontroller implementatie
  \end{outline}\\
  \hline
\end{tabularx}

\begin{tabularx}{\textwidth}{| l | X |}
  \hline
  Datum: & 30/03/2020 - 03/04/2020\\
  \hline
  Geplande taken: &
  \begin{outline}
    \1 opencv object detection, haarclassifier, HOG experimenten
    \1 YoloV3, tinyYoloV3
    \1 aanpassen dokerfiles
    \1 extra flags op gebieden voor meerdere mogelijke acties
    \1 tiny-yoloV3 optimizen
  \end{outline}\\
  \hline
  Stand van zaken: & 
  \begin{outline}
    \1 opencv object detection, haarclassifier, HOG werken niet geweldig
    \1 YoloV3 te zwaar om te runnen 
    \1 tinyYoloV3 werkt
    \1 extra flags op gebieden voor meerdere mogelijke acties
  \end{outline}\\
  \hline
  Problemen en knelpunten: & 
  \begin{outline}
    \1 tiny-yolo classified niet geweldig
  \end{outline}
  \\
  \hline
  Oplossingen: & 
  \begin{outline}
    \1 nog een andere classifier zoeken ?
    \1 stap terug zetten en verder kijken naar communicatie
  \end{outline}\\
  \hline
  Persoonlijke reflectie: & veel bijgeleerd over YoloV3, toch wel content over de vooruitgang van de code \\
  \hline
  Planning volgende week: & 
  \begin{outline}
    \1 gedurdende vakantie uitrusten (aan I-talent werken)
    \1 volgende werkweek projectomschrijving feedback van taal verwerken
    \1 paper schrijven + latex project aanmaken
  \end{outline}\\
  \hline
\end{tabularx}

\begin{tabularx}{\textwidth}{| l | X |}
  \hline
  Datum: & 20/04/2020 - 24/04/2020\\
  \hline
  Geplande taken: &
  \begin{outline}
    \1 Aan paper werken
    \1 taal feedback verwerken projectomschrijving
    \1 uitschrijven showcase
  \end{outline}\\
  \hline
  Stand van zaken: & 
  \begin{outline}
    \1 eerste draft paper is af
    \1 paper staat in latex project 
    \1 taalfeedback verwerkt voor projectomschrijving
    \1 showcase uitgeschreven , nu nog programmeren 
  \end{outline}\\
  \hline
  Problemen en knelpunten: & 
  \begin{outline}
    \1 paper schrijven was even puffen
  \end{outline}
  \\
  \hline
  Oplossingen: & 
  \begin{outline}
    \1 tanden bijten
  \end{outline}\\
  \hline
  Persoonlijke reflectie: & aan de paper werken was vermoeiender dan verwacht \\
  \hline
  Planning volgende week: & 
  \begin{outline}
    \1 showcase programmeren
  \end{outline}\\
  \hline
\end{tabularx}

\begin{tabularx}{\textwidth}{| l | X |}
  \hline
  Datum: & 27/04/2020 - 01/05/2020\\
  \hline
  Geplande taken: &
  \begin{outline}
    \1 final touches paper
    \1 taalversie afwerken 
    \1 showcases programmeren
    \1 vliegen op velocity 
  \end{outline}\\
  \hline
  Stand van zaken: & 
  \begin{outline}
    \1 taalversie paper is af
    \1 search functie bugs  
    \1 grote uitlijning voor de showcases ligt er 
  \end{outline}\\
  \hline
  Problemen en knelpunten: & 
  \begin{outline}
    \1 veel bugs geschreven en terug gevonden. Code was niet generiek genoeg.
  \end{outline}
  \\
  \hline
  Oplossingen: & 
  \begin{outline}
    \1 Veel opzoekwerk en nieuwe codeertechnieken gebruiken om de code te verbeteren.
  \end{outline}\\
  \hline
  Persoonlijke reflectie: & De oude code verbteren verloopt wat trager omdat het heet generiek moet zijn. \\
  \hline
  Planning volgende week: & 
  \begin{outline}
    \1 ombouwen naar vliegen op velocity
    \1 afwerken search functie
  \end{outline}\\
  \hline
\end{tabularx}


\begin{tabularx}{\textwidth}{| l | X |}
  \hline
  Datum: & 04/05/2020 - 08/05/2020\\
  \hline
  Geplande taken: &
  \begin{outline}
    \1 showcase afmaken 
    \1 vliegen op velocity ipv points
    \1 upgrade yolo 
    \1 coalitions uitbouwen
    \1 blackboard node upgrade voor object detection sharing knowledge 
    \1 object tracking
  \end{outline}\\
  \hline
  Stand van zaken: & 
  \begin{outline}
    \1 teams showcase is af
    \1 vliegen op velocity werkt
    \1 yolov4 werkt maar op CPU
    \1 coalitions werken nog niet
  \end{outline}\\
  \hline
  Problemen en knelpunten: & 
  \begin{outline}
    \1 OpenCV bug met yolo, detectie werkt enkel op CPU, dit is niet performant genoeg. Er staat ook een delay op (gevolge van performantie)
  \end{outline}
  \\
  \hline
  Oplossingen: & 
  \begin{outline}
    \1 onderzoek naar OpenCV builden van source
  \end{outline}\\
  \hline
  Persoonlijke reflectie: &  Goede vooruitgang geboekt. Code is generieker en makkelijker uitbreiden \\
  \hline
  Planning volgende week: & 
  \begin{outline}
    \1 Coalitions verder uitbouwen 
    \1 aan paper schrijven, feedback verwerken
    \1 data plotten
  \end{outline}\\
  \hline
\end{tabularx}

\begin{tabularx}{\textwidth}{| l | X |}
  \hline
  Datum: & 11/05/2020 - 15/05/2020\\
  \hline
  Geplande taken: &
  \begin{outline}
    \1 Docker containers aanpassen
    \1 Collision detection met andere drones
    \1 smoothing van het vliegen 
    \1 live plotting van data
  \end{outline}\\
  \hline
  Stand van zaken: & 
  \begin{outline}
    \1 Live plotten van data werkt
    \1 collision detection op andere thread
    \1 paper feedback verwerkt
  \end{outline}\\
  \hline
  Problemen en knelpunten: & 
  \begin{outline}
    \1 Live plotten van data zorgt voor interferance met het systeem
    \1 Collision detection is een zware thread, manier zoeken om dit wat te verlichten.
  \end{outline}
  \\
  \hline
  Oplossingen: & 
  \begin{outline}
    \1 Plotten van data naderhand laten gebeuren, met rosbag of ander medium.
  \end{outline}\\
  \hline
  Persoonlijke reflectie: & Goede vooruitgang paper geboekt, veel nieuwe kennis opgedaan \\
  \hline
  Planning volgende week: & 
  \begin{outline}
    \1 coalition afwerken en testen 
    \1 bespreken paper
    \1 blackboard node updaten voor generieke manier toevoegen van taken voor agents 
  \end{outline}\\
  \hline
\end{tabularx}

\begin{tabularx}{\textwidth}{| l | X |}
  \hline
  Datum: & 18/05/2020 - 22/05/2020\\
  \hline
  Geplande taken: &
  \begin{outline}
    \1 coalition afwerken 
    \1 coalition tests 
    \1 code refactoring 
    \1 bug fixes
    \1 blackboard node updaten voor generieke manier toevoegen van taken voor agents
  \end{outline}\\
  \hline
  Stand van zaken: & 
  \begin{outline}
    \1 blackboard node updaten voor generieke manier toevoegen van taken voor agents werkt
    \1 coalition af, niet getest
    \1 veel code herschreven
  \end{outline}\\
  \hline
  Problemen en knelpunten: & 
  \begin{outline}
    \1 coalition nog niet fatsoenlijk kunnen testen vanwege andere bugs
  \end{outline}
  \\
  \hline
  Oplossingen: & 
  \begin{outline}
    \1 code refactoring
    \1 bug fixes
  \end{outline}\\
  \hline
  Persoonlijke reflectie: &   veel bugs gevonden, nieuwe feature (generic task adding) is goed gelukt\\
  \hline
  Planning volgende week: & 
  \begin{outline}
    \1 coalition tests
    \1 performance tests
    \1 nieuwe docker image voor object detection on GPU
  \end{outline}\\
  \hline
\end{tabularx}

\begin{tabularx}{\textwidth}{| l | X |}
  \hline
  Datum: & 25/05/2020 - 29/05/2020\\
  \hline
  Geplande taken: &
  \begin{outline}
    \1 nieuwe docker image voor object detection on GPU
    \1 performance tests 
    \1 coalition tests 
    \1 team tests 
    \1 hologenic tests
  \end{outline}\\
  \hline
  Stand van zaken: & 
  \begin{outline}
    \1 nieuwe docker image aangemaakt 
    \1 OpenCV 4.3 van source gebuild 
    \1 hologenic test met 10 drones
  \end{outline}\\
  \hline
  Problemen en knelpunten: & 
  \begin{outline}
    \1 bug in openCV gevonden
    \1 met 10 drones moet de stepsize worden aangepast van de physics in de wereld, dit zorgt voor 'jiggling' drones 
  \end{outline}
  \\
  \hline
  Oplossingen: & 
  \begin{outline}
    \1 minder drones
    \1 over meerdere computers programma runnen
    \1 issue posten naar openCV
  \end{outline}\\
  \hline
  Persoonlijke reflectie: & Heel veel onderzoek gedaan naar openCV en heel veel bijgeleerd over builden van source. \\
  \hline
  Planning volgende week: & 
  \begin{outline}
    \1 afwerken tests
    \1 opnemen demos
    \1 data plots opslaan
  \end{outline}\\
  \hline
\end{tabularx}


\newpage
\subsection{Eindrapportage}

Ik heb gedurende deze stage enorm veel bijgeleerd. 


  % De eindrapportage wordt aan het einde van de stageperiode ingevuld. Hierin worden de volgende punten aangehaald:
    % • Opgedane ervaring
    % • Verloop van het project
    % • Gesignaleerde problemen
    % • Gekozen oplossing
    % • Persoonlijke reflectie
    % • Eindbesluit (het eindresultaat)
  % Dit is uiteindelijk ook je besluitvorming voor je eindwerk.