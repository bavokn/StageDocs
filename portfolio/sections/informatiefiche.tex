\begin{tabbing}
  Student: \textbf{\student}\\
  \\
  ~~~~~~ \= bknaeps@gmail.com\\
  \> +32 493 76 70 83\\
  \> AON - AI \& Robotics
\end{tabbing}
\begin{tabbing}
  Bedrijf stageplek: \textbf{\stagebedrijf}\\
  \\
  ~~~~~~ \= Elfde-liniestraat 24\\
  \> B-3500 Hasselt
\end{tabbing}
\begin{tabbing}
  Bedrijfspromotor: \textbf{Dhr. \bedrijfspromotor}\\
  \\
  ~~~~~~ \= +32 495 46 55 25\\
  \> tim.dupont@pxl.be
\end{tabbing}
\begin{tabbing}
  Hogeschoolpromotor: \textbf{Dhr. \hogeschoolpromotor}
\end{tabbing}
\clearpage 
Stageproject: \textbf{Navigating Multi Agent UAV systems in dynamic environments}
\begin{addmargin}[5ex]{0pt}
  
A single drone isn’t able to accomplish much on a large scale. Compare this to a bee trying to harvest a field of flowers, an impossible task for just a single bee. Nature solves this problem using swarm intelligence, using a colony of bees working together to solve a bigger problem. This phenomenon is not only perceived in bees. Birds flocking, hawks hunting, animal herding, bacterial growth, fish schooling and microbial intelligence are all examples of swarm intelligence in nature. Multi agent UAV systems are based on swarm intelligence, using multiple UAVs (aka drones) to solve more complex and bigger problems that are impossible for just a single drone to solve.

This research is conducted by PXL Smart ICT, the IT & Electronics center of expertise and part of the research department at PXL university of applied Sciences and Arts. The goal of the project is to enable IT-companies to implement UAV projects via rapid robot prototyping.

The internship provides a MAS architecture that supports homogeneous as well as heterogeneous structures. Each drone is an agent that processes data on its own and shares its findings with the other agents. The agent organization can handle hierarchical, hologenic and coalition teams. Different types of organizations will be compared regarding task completion time, computational time and accuracy of the completed task. A blackboard system takes care of knowledge sharing between agents, with the option of working over multiple computers. 

ROS (Robotic Operating System) is used to control the drone, a PX4 autopilot. It’s a set of software libraries and tools that help you build robot applications. The PX4 autopilot uses MAVLink, a very lightweight messaging protocol for communicating with drones. MAVROS, an implementation of MAVLink in ROS, is a component that can convert ROS topics and MAVLink messages allowing ArduPilot vehicles to communicate with ROS. This provides easy control of the UAVs. By gathering information from the UAV components the agents can analyze data. For example, the agent receives images from the camera and can detect if an object is a car using OpenCV, person or something else.

The goal of this internship is to create a flexible multi-agent system (MAS) architecture including a few exemplary demonstrators. In these examples, it’s made clear how each UAV makes decisions and how data is shared.

\end{addmargin}